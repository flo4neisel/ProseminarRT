\chapter{Konventionen und Beispiele}
Es folgen ein paar nützliche Beispiele und Konventionen.

\section{Gleichungen} \label{sec:gleichungen}
Bei umfangreichen Dokumenten sollte eine Gleichung, wie
%
\begin{equation} \label{equ:systemmatrix}
	\mathbf{A}_\mathrm{s} = \begin{bmatrix}1 & 0 \\ 0 & 2\end{bmatrix}
\end{equation}
%
nur eine Nummer bekommen, wenn sie, wie Gleichung~\eqref{equ:systemmatrix}, im Text referenziert wird. Die Nummerierung von Umgebungen kann über den * gesteuert werden.
% Absätze vor und nach Gleichungen vermeiden, um das Dokuemnt nicht unnötig auseinanderzuzerren

Gleichungen sind Teil des Satzes und bekommen deshalb auch Satzzeichen, wie beispielsweise die wahllosen Werte
%
\begin{align}
	\beta &= 5^{\circ}, \nonumber  \\
	c &= 42 \label{equ:antwort} 
\end{align}
%
und
\begin{equation*}
	\dot{x} = 7 \, \mathrm{\frac{m}{s}}.
\end{equation*}
Wie an Gleichung~\eqref{equ:antwort} zu sehen, können mehrere Gleichungen im Block ausgerichtet und einzeln nummeriert und referenziert werden.

\section{Bilder und Tabellen}
Alle Bilder und Tabelle sollten im Text referenziert werden.
Außer bei Fotos sollte auf Pixelgrafiken verzichtet werden. Vektorgrafiken können beispielsweise als PDF eingebunden werden, wie Abbildung~\ref{fig:rtm} zeigt.
Zur Erstellung von Skizzen eignet sich TikZ oder Inkscape mit der LaTex-Export-Funktion.

\begin{figure} %[h] die Figure-Umgebung bietet eine Option zur Steuerung der Platzierung
	\centering
	\includegraphics[width=0.4\textwidth]{rtm_mit_schrift.pdf}
	\caption{Regelungstechnik und Mechatronik}
	\label{fig:rtm}
\end{figure}

Wie in Tabelle~\ref{tab:tabelle} zu sehen, bekommen Tabellen eine Überschrift.
%
\begin{table}
	\centering
	\caption{Tabellenüberschrift}	
	\begin{tabular}{r|cc}
		& A & B \\
		\hline
		C & 1 & 2 \\
		D & 3 & 4 \\
	\end{tabular}
	\label{tab:tabelle}
\end{table}
%

\section{Blockschaltbilder}
Blockschaltbilder können mit der im Hauptdokument eingebundenen TikZ-Bibliothek erstellt werden.
%
\begin{center} % wird von LaTex nicht frei am Seitenanfang/Ende platziert. Im Normalfall aber besser figure nutzen.
%	\centering
	\input{Bilder/BSB_Beispiel.tikz}
	\captionof{figure}{TikZ Blockschaltbild in Figure-Umgebung}
	\label{fig:bsb_beispiel}
\end{center}

Variablen, wie $w$ aus Abbildung~\ref{fig:bsb_beispiel}, kommen auch im Fließtext immer in eine Mathe-Umgebung.

\section{Plots}
Plots können direkt in \LaTeX mit dem Paket \texttt{pgfplots} gesetzt werden.
So lassen sich Daten und Formatierung trennen und die Beschriftung der Plots fügt sich sehr gut ins Gesamtbild des Dokuments.
Aus Matlab heraus können Plots z.B. mit dem Tool \texttt{matlab2tikz} für \LaTeX{} konvertiert werden.
\texttt{pgfplots} kann aber auch direkt analytische Funktionen plotten: 
\tikz{	
	\begin{axis}[hide axis, height=18mm, width=5cm]
		\addplot[domain=-2*pi:2*pi, samples=100]{cos(deg(x))};
	\end{axis}
}.

\begin{figure}
	\centering
	% !TeX root = ../TUDaThesis.tex

% This file was created by matlab2tikz.
%
%The latest updates can be retrieved from
%  http://www.mathworks.com/matlabcentral/fileexchange/22022-matlab2tikz-matlab2tikz
%where you can also make suggestions and rate matlab2tikz.
%
\definecolor{mycolor1}{rgb}{0.00000,0.44700,0.74100}%
%
\begin{tikzpicture}

\begin{axis}[%
width=0.7\textwidth,
height=6cm,
scale only axis,
separate axis lines,
every outer x axis line/.append style={white!40!black},
every x tick label/.append style={font=\color{white!40!black}},
every x tick/.append style={white!40!black},
xmin=0,
xmax=14,
every outer y axis line/.append style={white!40!black},
every y tick label/.append style={font=\color{white!40!black}},
every y tick/.append style={white!40!black},
ymin=0,
ymax=2.5,
ylabel={Ausgang normiert},
xlabel={Zeit in Sekunden},
axis background/.style={fill=white}
]
\addplot [color=TUDa-2c, line width=1, forget plot]
  table[row sep=crcr]{%
0	0\\
0.0921034037197206	0.00411279805587128\\
0.184206807439441	0.0159482926069074\\
0.276310211159164	0.0347790289943219\\
0.368413614878884	0.0599132256263424\\
0.460517018598605	0.0906944372784881\\
0.552620422318325	0.126501104885605\\
0.644723826038048	0.166746003636801\\
0.736827229757768	0.210875600557069\\
0.828930633477489	0.258369332141982\\
0.921034037197209	0.308738812005936\\
1.01313744091693	0.361526977911547\\
1.10524084463665	0.416307186968996\\
1.19734424835637	0.472682267230876\\
1.38155105579581	0.588769773524355\\
1.93417147811414	0.942361883863663\\
2.11837828555358	1.05580915778143\\
2.2104816892733	1.11091852101091\\
2.30258509299302	1.16478018774458\\
2.39468849671275	1.21728576867211\\
2.48679190043247	1.26834190455422\\
2.57889530415219	1.31786932190199\\
2.67099870787191	1.36580191469028\\
2.76310211159163	1.41208585353384\\
2.85520551531135	1.45667872351849\\
2.94730891903107	1.49954869165797\\
3.03941232275079	1.54067370474229\\
3.13151572647051	1.5800407181526\\
3.22361913019023	1.61764495604156\\
3.31572253390996	1.65348920311623\\
3.40782593762968	1.68758312811146\\
3.4999293413494	1.71994263890565\\
3.59203274506912	1.75058926910707\\
3.68413614878884	1.77954959582612\\
3.77623955250856	1.80685468824809\\
3.86834295622828	1.83253958652982\\
3.960446359948	1.85664281046295\\
4.05254976366772	1.87920589727478\\
4.14465316738744	1.90027296787515\\
4.23675657110716	1.91989032080341\\
4.32885997482689	1.93810605308292\\
4.42096337854661	1.95496970715117\\
4.51306678226633	1.97053194300157\\
4.60517018598605	1.98484423464626\\
4.69727358970577	1.99795858998971\\
4.78937699342549	2.00992729318779\\
4.88148039714521	2.02080266855745\\
4.97358380086493	2.03063686509711\\
5.15779060830437	2.04738828496254\\
5.34199741574382	2.06058825850954\\
5.52620422318326	2.07063002123016\\
5.7104110306227	2.07788951993463\\
5.89461783806214	2.08272250421635\\
6.07882464550158	2.08546241975432\\
6.26303145294103	2.08641900115247\\
6.44723826038047	2.08587746840164\\
6.72354847153963	2.08281911140062\\
6.99985868269879	2.07774554823536\\
7.36827229757768	2.06896203894915\\
8.19720293105517	2.04625070133987\\
8.74982335337349	2.03209506210472\\
9.2103403719721	2.02202542105953\\
9.6708573905707	2.0138167145694\\
10.1313744091693	2.00747359450982\\
10.5918914277679	2.00284185840762\\
11.1445118500862	1.99919994883539\\
11.7892356761243	1.99698913404195\\
12.6181663096018	1.99626757654961\\
13.9997173653976	1.99742653315299\\
14.0918207691173	1.99753717612361\\
};
\addplot [color=black, dotted, forget plot]
  table[row sep=crcr]{%
-1	2\\
15.4	2\\
};
\end{axis}

\end{tikzpicture}%
	\caption{Sprungantwort $\mathrm{PT}_2$}
	\label{fig:step_pt2}
\end{figure}

Man sollte darauf achten, dass nicht unnötig viele Datenpunkte geplottet werden. Das verlangsamt das Kompilieren und bläht die Dateigröße des pdf auf.
\texttt{matlab2tikz} liefert dafür z.B. \texttt{cleanfigure} mit, welches auch gleich seltsame Achsenskalierungen repariert.
Hat man viele und komplexe Plots, bietet sich die \texttt{external}-Bibliothek von \texttt{TikZ} an.

Für den perfekten Gesamteindruck nutzt man die offizielle TU-Farbpalette auch für die Plots.
Die TU-Farben liefert das Paket im Format \texttt{TUDa-xy} mit.
In Abbildung~\ref{fig:step_pt2} wird z.B. die rtm-Farbe \textcolor{TUDa-2c}{\texttt{TUDa-2c}} genutzt.
Für beste Lesbarkeit auch bei einem Schwarzweißdruck oder Farbenblindheit sollten sich geplottete Linien nicht nur farblich, sondern auch durch den Linienstil unterscheiden (\texttt{dotted}, \texttt{dashed}, ...).

