\section*{Aufgabenstellung}
In diesem Projektseminar sollen die nichtlinearen Bewegungsgleichungen eines Helikopters hergeleitet und zum Entwurf einer Positionsregelung verwendet werden. Hierzu sind die in der Flugmechanik geläufigen Koordinatensysteme und Notationen zu verwenden. Zudem ist auf die Besonderheiten von Helikoptersystemen zu achten, die sich stark in der Auftriebsgenerierung sowie der Steuerung äußern.

Zunächst ist auf Basis des Modells ein Reglerentwurf durchzuführen, der das Fluggerät stabil auf einer einstellbaren Position einregelt. Anschließend soll die Regelung derart erweitern werden, dass eine am Helikopter befestigte Last (bspw. eine Rettungstrage) positionsgeregelt wird. In beiden Fällen sind das Schwingungs- und das Störverhalten des geregelten Systems zu berücksichtigen.
Das Modell und die Regelung sind in \textsc{Matlab-Simulink} zu implementieren und zu evaluieren. Für die Implementierung ist die Parametrierung eines Beispiel-Helikopters zu recherchieren und zu übernehmen.

Sämtliche Ergebnisse sind ausführlich zu dokumentieren.

\SADAAufgabenstellung